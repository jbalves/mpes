\documentclass[12pt,a4paper]{article}
\usepackage[utf8]{inputenc}
\usepackage[brazil]{babel}
\usepackage[T1]{fontenc}
\usepackage{makeidx}

\begin{document}

%------------------
% CAPA
%------------------
\begin{titlepage}
%\begin{center}{\bf \Large Projeto de Pesquisa}\\[3.5cm]
%\end{center}
\begin{flushright}
{\bf CESAR SCHOOL}\\
{Unidade de Educação do Centro de Estudos e Sistemas Avançados do Recife}
{(CESAR)}\\[3.5cm]
{\bf \large MESTRADO PROFISSIONAL em Engenharia de Software}\\
{Proposta de Pesquisa}\\[2.5cm] 
{\bf Autor:}\\
{Jeferson Barros Alves}\\[0.8cm]
%{\bf Orientador:}\\
%{nome do orientador}\\[0.8cm]
{\bf Linha de Pesquisa:}\\
{Fábrica de Software}\\[0.8cm]
\end{flushright}
\end{titlepage}
%---------------------
 

\tableofcontents
\addtocontents{toc}{\protect\thispagestyle{empty}}
\newpage

%------------------
% APRESENTAÇÃO
%------------------
\section{Apresentação}
%Qual o objetivo deste documento, em um parágrafo.
Este documento tem como objetivo propor um projeto de estudo a ser desenvolvido durante o Mestrado Profissional em Engenharia de Software, com intuito de trabalhar com conteúdos práticos de necessidade do autor, adquirindo como resultado uma evolução nas experiências na área de atuação profissional e construção de competências que agregam maior valor profissional.

%Observações:
%\begin{itemize}
%\item As seções 2 a 5 devem somar no mínimo 2 (duas) e no máximo 5 (cinco) páginas.
%\end{itemize}
%------------------
% MOTIVAÇÃO
%------------------
\section{Motivação para a pesquisa}
%Identificando um problema a ser resolvido, \cite{Mazo2012} por que o problema existe e qual sua relevância
%no contexto do Mestrado Profissional em Engenharia de Software?
Em empresas com alta demanda de produção de software,pode-se observar uma busca por redução de custos, tempo de produção, melhoria na qualidade, manutenabilidade de código, dentre outros fatores. Uma da estratégia adota pelas empresas para alncançar tais objetivos é a reautilização de software.

De acordo com \cite{sametinger1997}, o reuso de software foi introduzido com uma das alternativas para superar a crise de software, termo criado para descrever o aumento da carga de frustração do desenvolvimento e manutenção de softwares. Impactando positivamente em aspectos como qualidade, custos e produtividade.

Segundo \cite{Sommerville2011}, a disponibilidade de softwares reusáveis tem aumentado significativamente. Algumas grandes empresas fornecem uma variedade de componentes reutilizáveis para seus clientes. Padrões, como \textit{web service}, tornaram mais fácil o desenvolvimento de serviços gerais e reúso destes em uma variedade de aplicações.

Ao logo dos últimos vinte anos, muita técnicas foram desenvolvidas para oferecer suporte a reúso de software. Essa técnicas exploram os fatos de os sistemas, no mesmo domínio de aplicação, serem semelhantes e terem potencial para reúso. O reúso é possível em diferentes níveis, desde funções simples até aplicações completas, e normas para componentes reusáveis facilitam o reúso \cite{Sommerville2011}. Dentre essas técnicas podemos salientar a linha de produto de software, que sengundo \cite{Sommerville2011} é um tipo de aplicação é generalizado em torno de uma arquitetura comum para que esta possa ser adaptada para diferentes clientes.

%------------------
% PROBLEMA
%------------------
\section{Descrição do problema a ser resolvido}
%Qual o problema que você pretende investigar/resolver?
O teste de software muitas vezes requer mais trabalho de projeto do que qualquer outra ação da engenharia de software. Se for feito casualmente, perde-se tempo, fazem-se esforços desnecessários, e, ainda pior, erros passam sem ser detectados. Portanto, é razoável estabelecer uma estratégia sistemática para teste de software \cite{pressman2011}.

O objetivo principal desta proposta é a instanciação e automatização de um processo de Linha de Produto de Software orientado a família de produtos no domínio de testes automatizados para o sistema operacional \textit{Android}.
%------------------
% METODOLOGIA
%------------------
\section{Metodologia}
%Como você pretende resolver o problema em questão?
Uma das abordagens mais eficazes para o reúso é criar linhas de produto de software ou famílias de aplicações. Uma linha de produto de software é um conjunto de aplicações com uma arquitetura comum e componentes compartilhados, sendo cada aplicação especializada para refletir necessidades diferentes. O núcleo do sistema é projetado para ser configurado e adaptado para atender às necessidades de clientes diferentes. Isso pode envolver a configuração de alguns componentes, implementação de componentes adicionais e a modificação de alguns dos componentes para refletir novos requisitos \cite{Sommerville2011}.

%------------------
% CONCLUSÕES
%------------------
\section{Conclusões e considerações finais}
Conclusão da proposta de pesquisa.

\newpage
%------------------
% BIBLIOGRAFIA
%------------------
\bibliographystyle{abnt-alf}
\bibliography{projeto_pesquisa}
\end{document}


