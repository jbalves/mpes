\documentclass[12pt,a4paper]{article}
\usepackage[utf8]{inputenc}
\usepackage[brazil]{babel}
\usepackage[T1]{fontenc}
\usepackage{makeidx}
\usepackage{tocloft}

\begin{document}
%------------------
% CAPA
%------------------
\begin{titlepage}
\begin{flushright}
{\bf CESAR SCHOOL}\\
{Unidade de Educação do Centro de Estudos e Sistemas Avançados do Recife}
{(CESAR)}\\[3.5cm]
{\bf \large MESTRADO PROFISSIONAL em Engenharia de Software}\\
{Proposta de Pesquisa}\\[2.5cm] 
{\bf Autor:}\\
{Jeferson Barros Alves}\\[0.8cm]
%{\bf Orientador:}\\
%{nome do orientador}\\[0.8cm]
{\bf Linha de Pesquisa:}\\
{Fábrica de Software}\\[0.8cm]
\end{flushright}
\end{titlepage}
%---------------------
\renewcommand{\cftsecleader}{\cftdotfill{\cftdotsep}} %Whether you want dots on table of contents
\tableofcontents
\addtocontents{toc}{\protect\thispagestyle{empty}}
\newpage

%------------------
% APRESENTAÇÃO
%------------------
\section{Apresentação}
%Qual o objetivo deste documento, em um parágrafo.
Este documento tem como objetivo propor um projeto de estudo a ser desenvolvido durante o Mestrado Profissional em Engenharia de Software, com intuito de trabalhar com conteúdos práticos de necessidade do autor, adquirindo como resultado uma evolução nas experiências na área de atuação profissional e construção de competências que agregam maior valor profissional.

O objetivo principal desta proposta é a instanciação de um processo de Linha de Produto de Software orientado a família de produtos no domínio de testes automatizados para o sistema operacional \textit{Android}.

%Observações:
%\begin{itemize}
%\item As seções 2 a 5 devem somar no mínimo 2 (duas) e no máximo 5 (cinco) páginas.
%\end{itemize}
%------------------
% MOTIVAÇÃO
%------------------
\section{Motivação para a pesquisa}
%Identificando um problema a ser resolvido, \cite{Mazo2012} por que o problema existe e qual sua relevância
%no contexto do Mestrado Profissional em Engenharia de Software?
Em empresas com alta demanda de produção de software,pode-se observar uma busca por redução de custos, tempo de produção, melhoria na qualidade, manutenabilidade de código, dentre outros fatores. Uma da estratégia adota pelas empresas para alncançar tais objetivos é a reautilização de software.

De acordo com \cite{sametinger1997}, o reuso de software foi introduzido com uma das alternativas para superar a crise de software, termo criado para descrever o aumento da carga de frustração do desenvolvimento e manutenção de softwares. Impactando positivamente em aspectos como qualidade, custos e produtividade.

Segundo \cite{Sommerville2011}, a disponibilidade de softwares reusáveis tem aumentado significativamente. Algumas grandes empresas fornecem uma variedade de componentes reutilizáveis para seus clientes. Padrões, como \textit{web service}, tornaram mais fácil o desenvolvimento de serviços gerais e reúso destes em uma variedade de aplicações.

Ao logo dos últimos vinte anos, muita técnicas foram desenvolvidas para oferecer suporte a reúso de software. Essa técnicas exploram os fatos de os sistemas, no mesmo domínio de aplicação, serem semelhantes e terem potencial para reúso. O reúso é possível em diferentes níveis, desde funções simples até aplicações completas, e normas para componentes reusáveis facilitam o reúso \cite{Sommerville2011}. Dentre essas técnicas podemos salientar a linha de produto de software, que sengundo \cite{Sommerville2011} é um tipo de aplicação é generalizado em torno de uma arquitetura comum para que esta possa ser adaptada para diferentes clientes.

Em \cite{orozco2008}, foi proposto uma aglutinação dos requisitos comuns das abordagens de teste baseados em modelos através de uma arquitetura de linha de produtos de software, proporcionando assim o reuso dos elementos arquiteturais comuns destas abordagens, facilitando a implementação de ferramentas, diminuindo o tempo de desenvolvimento e aumentando a qualidade destas ferrasmentas, sendo estas vantagens promovidas pelos conceitos de linha de produtos de software.

No trabalho de \cite{pontes2017}, observamos que é proposto um método de extração e evolução de linha de produto de software a partir de sistemas existentes implementados na linguagem Java no domínio de sistemas de controle de espaço físico atendendo novos requisitos durante o processo de evolução.

%------------------
% PROBLEMA
%------------------
\section{Descrição do problema a ser resolvido}
%Qual o problema que você pretende investigar/resolver?
O teste de software muitas vezes requer mais trabalho de projeto do que qualquer outra ação da engenharia de software. Se for feito casualmente, perde-se tempo, fazem-se esforços desnecessários, e, ainda pior, erros passam sem ser detectados. Portanto, é razoável estabelecer uma estratégia sistemática para teste de software \cite{pressman2011}.

De acordo com \cite{bertolino2007} teste é uma atividade essencial em Engenharia de Software. Em termos, simples é a observação da execução de um sistema de software para validar se a execução ocorre como o esperado e identifica potenciais defeitos. Testes são amplamente usados na indústria para medida de qualidade.

A automação de teste de software pode reduzir drasticamente o esforço requerido para as atividades de teste. Através da automação, os testes podem ser realizados em minutos ao invés de demorarem horas para serem executados manualmente, podendo alcançar uma diminuição de esforço em mais de 80\%  \cite{fewster1999}.
%------------------
% METODOLOGIA
%------------------
\section{Metodologia}
%Como você pretende resolver o problema em questão?
Uma das abordagens mais eficazes para o reúso é criar linhas de produto de software ou famílias de aplicações. Uma linha de produto de software é um conjunto de aplicações com uma arquitetura comum e componentes compartilhados, sendo cada aplicação especializada para refletir necessidades diferentes. O núcleo do sistema é projetado para ser configurado e adaptado para atender às necessidades de clientes diferentes. Isso pode envolver a configuração de alguns componentes, implementação de componentes adicionais e a modificação de alguns dos componentes para refletir novos requisitos \cite{Sommerville2011}.

Será necessário realizar uma reengenharia do software no projeto de teste automatizado existente como base para a construção um sistema de produtos de software. Como mostrado em \cite{assunccao2017}, a reengenharia para a criação de uma sistema de produtos de software pode ser classificada em três fases: Detecção, sendo o inicio do processo, baseada na detecção de características comuns e variáveis dos artefatos; Análise e a organização das características, voltada para a criação de um modelo de variabilidade; Transformação, onde os artefatos são utilizados para a criação do sistema de produto de software.

Primeiramente será realizado um levantamento bibligráfico dos trabalhos que tenham como tema o desenvolvimento de testes automatizados e linha de produtos de software como forma de fortalecer o conceito, em seguida um segundo levantamento será realizado sobre trabalhos que demonstram uma integração desses dois conteúdos.

%------------------
% CONCLUSÕES
%------------------
\section{Conclusões e considerações finais}
Pode-ser observar que uma linha de produtos de software, geralmente é iniciada a partir de um conjunto de sistemas já em produção que passam por um processo de reengenharia para atender a demanda deste contexto.

Este trabalho buscar relacionar testes automatizados com linha de produtos de software, com o intuito de facilitar a elaboração de produtos aplicando instâncias para versões do sistema operacional \textit{Android}, assim como verificar sua eficâcia no ambiente de produção.


\newpage
%------------------
% BIBLIOGRAFIA
%------------------
\bibliographystyle{abnt-alf}
\bibliography{projeto_pesquisa}
\end{document}


