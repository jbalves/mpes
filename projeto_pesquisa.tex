\documentclass[10pt,a4paper]{article}
\usepackage[utf8]{inputenc}
\usepackage[brazil]{babel}
\usepackage[T1]{fontenc}

\begin{document}

%------------------
% CAPA
%------------------
\begin{titlepage}
\begin{center}{\bf \Large Projeto de Pesquisa}\\[3.5cm]
\end{center}
\begin{flushright}
{\bf CESAR SCHOOL}\\
{Unidade de Educação do Centro de Estudos e Sistemas Avançados do Recife}
{(CESAR)}\\[3.5cm]
{\bf \large MESTRADO PROFISSIONAL em Engenharia de Software}\\
{Proposta de Pesquisa}\\[2.5cm] 
{\bf Autor:}\\
{Jeferson Barros Alves}\\[0.8cm]
%{\bf Orientador:}\\
%{nome do orientador}\\[0.8cm]
{\bf Linha de Pesquisa:}\\
{Fábrica de Software}\\[0.8cm]
\end{flushright}
\end{titlepage}
%---------------------
 
\tableofcontents

\newpage

%------------------
% APRESENTAÇÃO
%------------------
\section{Apresentação}
Qual o objetivo deste documento, em um parágrafo.
Este documento tem como objetivo propor um projeto de estudo a ser desenvolvido durante o Mestrado Profissional em Engenharia de Software, com intuito de trabalhar com conteúdos práticos de necessidade do autor, adquirindo como resultado uma evolução nas experiências na área de atuação profissional e construção de competências que agregam maior valor profissional.

%Observações:
%\begin{itemize}
%\item As seções 2 a 5 devem somar no mínimo 2 (duas) e no máximo 5 (cinco) páginas.
%\end{itemize}
%------------------
% MOTIVAÇÃO
%------------------
\section{Motivação para a pesquisa}
Identificando um problema a ser resolvido, \cite{Mazo2012} por que o problema existe e qual sua relevância
no contexto do Mestrado Profissional em Engenharia de Software?
Segundo \cite{Sommerville2011}, a disponibilidade de softwares reusáveis tem aumentado significativamente. Algumas grandes empresas fornecem uma variedade de componentes reutilizáveis para seus clientes. Padrões, como \textit{web service}, tornaram mais fácil o desenvolvimento de serviços gerais e reúso destes em uma variedade de aplicações

%------------------
% PROBLEMA
%------------------
\section{Descrição do problema a ser resolvido}
Qual o problema que você pretende investigar/resolver?

%------------------
% METODOLOGIA
%------------------
\section{Metodologia}
Como você pretende resolver o problema em questão?

%------------------
% CONCLUSÕES
%------------------
\section{Conclusões e considerações finais}
Conclusão da proposta de pesquisa.


%------------------
% BIBLIOGRAFIA
%------------------
\bibliographystyle{abnt-alf}
\bibliography{projeto_pesquisa}
\end{document}


